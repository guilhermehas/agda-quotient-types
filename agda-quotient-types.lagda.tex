\documentclass{beamer}

% Use the input encoding UTF-8 and the font encoding T1.
\usepackage[utf8]{inputenc}
\usepackage[T1]{fontenc}

% Support for Agda code.
\usepackage{agda}

% Decrease the indentation of code.
\setlength{\mathindent}{1em}

% Customised setup for certain characters.
\usepackage{newunicodechar}
\newunicodechar{∀}{\ensuremath{\mathnormal{\forall}}}
\newunicodechar{→}{\ensuremath{\mathnormal{\to}}}
\newunicodechar{≡}{\ensuremath{\mathnormal{\equiv}}}
\newunicodechar{ℓ}{\ensuremath{\mathnormal{\ell}}}
\newunicodechar{∣}{\ensuremath{\mathnormal{\|}}}
\newunicodechar{∥}{\ensuremath{\mathnormal{\||}}}
\newunicodechar{₀}{\ensuremath{{}_0}}
\newunicodechar{₁}{\ensuremath{{}_1}}
\newunicodechar{₂}{\ensuremath{{}_2}}
\newunicodechar{ℚ}{\ensuremath{\mathnormal{\mathbb{Q}}}}
\newunicodechar{ℤ}{\ensuremath{\mathnormal{\mathbb{Z}}}}
\newunicodechar{ℕ}{\ensuremath{\mathnormal{\mathbb{N}}}}
\newunicodechar{×}{\ensuremath{\mathnormal{\times}}}
\newunicodechar{∷}{\ensuremath{::}}

% Support for Greek letters.
\usepackage{alphabeta}

% Disable ligatures that start with '-'. Note that this affects the
% entire document!
\usepackage{microtype}
\DisableLigatures[-]{encoding=T1}

%Information to be included in the title page:
\title{Using quotient types in Agda}
\author{Guilherme Silva}
\date{2021}


\begin{document}

\frame{\titlepage}

\begin{frame}
  \frametitle{Motivation}
  In a simple type theory without quotient types,
  it is not possible to have two distinct elements and after
  making them equal.

  With them, it is now possible.
  One of the best examples is making $ \frac{1}{2} $ and $ \frac{2}{4} $ be equal elements.
\end{frame}

\begin{frame}
  \frametitle{Imports}
  \begin{code}
  {-# OPTIONS --cubical #-}

  open import Cubical.Foundations.Prelude hiding (isProp; isSet)
  open import Cubical.Relation.Nullary using (¬_)

  open import Cubical.Data.Unit
  open import Cubical.Data.Bool
  open import Cubical.Data.Int
  open import Cubical.Data.Nat hiding (_·_)

  private variable
    ℓ ℓ₁ : Level
    A : Set ℓ
    B : Set ℓ₁
    x y z : A
  \end{code}
\end{frame}

\begin{frame}
  \frametitle{Equalities}
  In cubical type theory, equality is a function between an interval to a set.
  \begin{code}
  tt≡tt : tt ≡ tt
  tt≡tt i = tt

  refl' : (x : A) → x ≡ x
  refl' x i = x
  \end{code}

\end{frame}

\begin{frame}
  \frametitle{Equalities}
  It is also possible to invert the path of equality.
  If the interval i goes from i0 to i1, the interval \~{} i goes from i1 to i0.

  \begin{code}
  sym' : x ≡ y → y ≡ x
  sym' p i = p (~ i)

  ext : (f g : A → B) → (∀ x → f x ≡ g x) → f ≡ g
  ext f g p i x = -- wants that when i = i0, f x
                  --        and when i = i1, g x
                  p x -- is the same of (f x ≡ g x)
                  i
  \end{code}

\end{frame}

\begin{frame}
  \frametitle{The simplest example of quotient types}
  This is an example when two elements of the same data types are equal:
  \begin{code}
  data Bool≡ : Set where
    true false : Bool≡
    t≡f : true ≡ false

  _ : true ≡ false
  _ = t≡f
  \end{code}
\end{frame}

\begin{frame}
  \frametitle{Functions of quotient types}
  Let a function f between elements of Bool≡ and another set, \\
  if $ x \equiv y $, then $ f(x) \equiv f(y) $.
  In this case, $ x $ is true, $ y $ is false and $ f(x) $ is tt.
  \begin{code}
  f : Bool≡ → Unit
  f true    = tt
  f false   = tt
  f (t≡f i) = refl i -- proving that f true ≡ f false

  \end{code}

\end{frame}

\begin{frame}
  \frametitle{Truncation type}

  In truncation type, every element of it is equal.
  With that, we can define Bool≡ as ∥ Bool ∥.

  \begin{code}
  data ∥_∥ {ℓ} (A : Set ℓ) : Set ℓ where
    ∣_∣ : A → ∥ A ∥
    squash : ∀ (x y : ∥ A ∥) → x ≡ y

  Bool≡' = ∥ Bool ∥
  \end{code}
\end{frame}

\begin{frame}
  \frametitle{Not all equalities are the same}
  In cubical type theory, the equalities are not always the same.
  In this example, the circle is isomorphic (the same) to the integers:
  \begin{code}
  data Circle : Type where
    base : Circle
    loop : base ≡ base
  \end{code}
\end{frame}

\begin{frame}
  \frametitle{Making equalities equal}
  To make the equality equal, it is necessary to give the information in the data type that it is a set:
  \begin{code}
  isProp : Type ℓ → Type ℓ
  isProp A = (x y : A) → x ≡ y

  isSet : Type ℓ → Type ℓ
  isSet A = (x y : A) → isProp (x ≡ y)
  \end{code}
\end{frame}

\begin{frame}
  \frametitle{Rational Numbers}
  The rational numbers are defined giving the numerator and a denominator different from zero.
  Two rational numbers $ \frac{p₁}{q₁} $ and $ \frac{p₂}{q₂} $ are equal when $ p₁ \times q₂ \equiv p₂ \times q₁ $
  \begin{code}
  data ℚ : Type where
    _/_[_] : (p : ℤ) (q : ℤ) → ¬ (q ≡ pos 0) → ℚ
    path : ∀ p₁ q₁ p₂ q₂ {pr₁ pr₂} → (p₁ · q₂) ≡ (p₂ · q₁)
      → p₁ / q₁ [ pr₁ ] ≡ p₂ / q₂ [ pr₂ ]
    trunc : isSet ℚ

  _ : ∀ {pr₁ : ¬ (2 ≡ pos 0)} {pr₂ : ¬ (4 ≡ pos 0)}
    → 1 / 2 [ pr₁ ] ≡ 2 / 4 [ pr₂ ]
  _ = path 1 2 2 4 refl
  \end{code}
\end{frame}

\begin{frame}
  \frametitle{Integers}
  Another way to define integers is making the positive and negative zero equals.
  \begin{code}
  data ℤ' : Type where
    pos' : ℕ → ℤ'
    neg' : ℕ → ℤ'
    0+-≡ : pos' 0 ≡ neg' 0
  \end{code}
\end{frame}

\begin{frame}
  \frametitle{Sets}
  One way of defining sets is to create a list where you can swap elements and remove when they are equal.
  \begin{code}
  infixr 20 _∷_
  data LFSet (A : Type ℓ) : Type ℓ where
    []    : LFSet A
    _∷_   : (x : A) → (xs : LFSet A) → LFSet A
    dup   : ∀ x xs   → x ∷ x ∷ xs ≡ x ∷ xs
    comm  : ∀ x y xs → x ∷ y ∷ xs ≡ y ∷ x ∷ xs
    trunc : isSet (LFSet A)

  _ : tt ∷ tt ∷ [] ≡ tt ∷ []
  _ = dup tt []

  _ : let true∷false : LFSet Bool
          true∷false = true ∷ false ∷ []
      in true∷false ≡ false ∷ true ∷ []
  _ = comm true false []
  \end{code}
\end{frame}

\begin{frame}
  \vspace*{36 pt}
  \begin{center}
  {\Huge Any questions?}
  \end{center}
\end{frame}


\end{document}
